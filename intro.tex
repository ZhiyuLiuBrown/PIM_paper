%\section{Introduction}
\section{Near-memory Computing}

The performance gap between memory and CPU has grown exponentially. Memory vendors have focused mostly on improving memory capacity and bandwidth, sometimes even at the cost of increased memory access latencies. To provide higher bandwidth with lower access latencies, hardware architects have proposed near-memory computing (NMC), where a lightweight processor is located close to memory. [details?] This is an old idea, that has been intensely studied in the past~\cite{}, but so far has not materialized. However, new advances in 3D integration and in die stacked memory make near-memory computing viable in the near future. [details?] This new technology promises to revolutionize the interaction between compute and data, as memory becomes an active component in managing the data. Therefore, it invites a fundamental rethinking of basic data structures and promotes a tighter dependency between algorithmic design and hardware characteristics. 

But how do we design and optimize data structures for near-memory computing? And how do these algorithms compare to traditional CPU-managed concurrent data structures? To answer these questions, we develop a simplified model of the expected performance of NMC using the flat combining technique~\cite{}. Using this model, we show that a naive NMC-managed data structure cannot outperform a carefully crafted CPU-managed \emph{concurrent} data structure. In particular, the lower latency access to memory cannot compensate for the loss of parallelism. Server machines nowadays have hundreds of cores; algorithms for concurrent data structures exploit these cores to achieve high throughput and scalability. Therefore, to be competitive with traditional concurrent data structures, NMC data structures need new approaches to leverage parallelism. We investigate two classes of data structures. First, pointer chasing data structures, which have a high degree of inherent parallelism. We propose using techniques such as combining~\cite{} and partitioning the data across vaults to reintroduce parallelism for these data structures. Secondly, we evaluate contended data structures, such as FIFO queues, which are more likely to have the data in hardware caches, therefore not benefiting that much from NMC. We design a new NMC FIFO queue that exploits pipelining to outperform prior concurrent FIFO queues, despite their high locality benefit from caches. 

%TODO
Prior work has focused on using the increased bandwidth; the work that focused on the reduced latency ignored parallelism. For example, cite [] has shown that pointer chasing data structures can benefit from the shorter latency access, but did not evaluate the performance loss from the loss of parallelism. 

%TODO
%Closing 