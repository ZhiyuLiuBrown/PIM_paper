%\section{Introduction}
\section{Near-memory Computing}

The performance gap between memory and CPU has grown exponentially.
Memory vendors have focused mainly on improving memory capacity and
bandwidth, sometimes even at the cost of higher memory access
latencies \cite{Chang:2016, DBLP:conf/hpca/LeeKSLSM13,
  DBLP:conf/hpca/LeeKPKSCM15, Chang:2017, Lee:2017, Mutlu:2013}.  To
provide higher bandwidth with lower access latencies, hardware
architects have proposed near-memory computing (also called
\textit{processing-in-memory}, or PIM), where a lightweight processor
(called a PIM core) is located close to memory. A memory access from a
PIM core is much faster than that from a CPU core.  Near-memory
computing is an old idea that has been intensively studied in the past
(e.g., \cite{Stone1970, Kogge1994, Gokhale1995, Patterson1997,
  Oskin1998, KangHYKGLTP99, Hall1999, Elliott:1992}), but so far has
not yet materialized. However, new advances in 3D integration and
die-stacked memory likely make near-memory computing viable in the
near future.  For example, one PIM design \cite{Ahn2015:2,
  Zhang2014:TTP, Ahn2015:1, boroumand2016} assumes that memory is
organized in multiple vaults, each having an in-order PIM core to
manage it.  These PIM cores can communicate through message passing,
but do not share memory, and cannot access each other's vaults.

This new technology promises to revolutionize the interaction between
computation and data, as it enables memory to become an active
component in managing the data.  Therefore, it invites a fundamental
rethinking of basic data structures and promotes a tighter dependency
between algorithmic design and hardware characteristics.

Prior work has already shown significant performance improvements by
using PIM for embarrassingly parallel and data-intensive
applications~\cite{Zhang2014:TTP, Ahn2015:2, ZhuASSHPF13,
  Akin2015:DRM, Hsieh:2016:TOM}, as well as for pointer-chasing
traversals~\cite{hsieh2016accelerating,Hashemi:2016} in
\emph{sequential} data structures.  However, current server machines
have hundreds of cores, and algorithms for concurrent data structures
exploit these cores to achieve high throughput and scalability, with
significant benefits over sequential data structures (e.g.,
\cite{practicallf, skiplists-concpugh, valois, Herlihy08}).  Unlike
prior work, we focus on \emph{concurrent} data structures for PIM and
we show that naive PIM data structures \emph{cannot} outperform
state-of-the-art concurrent data structures.  In particular, the
lower-latency access to memory provided by PIM cannot compensate for
the loss of parallelism in data structure manipulation.  For example,
we show that even if a PIM memory access is two times faster than a CPU memory
access, a sequential PIM linked-list is still slower than a
traditional concurrent linked-list accessed in parallel by only three
CPU cores.

Therefore, to be competitive with traditional concurrent data
structures, PIM data structures need new algorithms and new approaches
to leverage parallelism.  As the PIM technology approaches fruition,
it is crucial to investigate how to best utilize it to exploit the
lower latencies, while still leveraging the vast amount of previous
research related to concurrent data structures.

In this paper, we provide answers to the following key questions: 1)
How do we design and optimize data structures for PIM? 2) How do these
optimized PIM data structures compare to traditional CPU-managed
concurrent data structures? To answer these questions, even before the
hardware becomes available, we develop a simplified model of the
expected performance of PIM. Using this model, we investigate two
classes of data structures.

First, we analyze \emph{pointer chasing data structures} (Section
\ref{section:pointer_chasing}), which have a high degree of inherent
parallelism and low contention, but which incur significant overhead
due to hard-to-predict memory access patterns.  We propose using
techniques such as combining and partitioning the data across vaults
to reintroduce parallelism for these data structures.

Second, we explore \emph{contended data structures} (Section
\ref{section:contended}), such as FIFO queues, which can leverage CPU
caches to exploit their inherent high locality.  As they exploit the
fast on-chip caches well, FIFO queues might not seem to be a good fit
for leveraging PIM's faster memory accesses.  Nevertheless, these data
structures exhibit a high degree of contention, which makes it
difficult, even for the most advanced data structures, to obtain good
performance when many threads access the data concurrently.  We use
pipelining of requests, which can be done very efficiently in PIM, to
design a new FIFO queue suitable for PIM that can outperform
state-of-the-art concurrent FIFO queues~\cite{Morrison13, Hendler10}.

The contributions of this paper are as follows:
\begin{itemize}
\item We propose a simple and intuitive model to analyze the
  performance of PIM data structures and of concurrent data
  structures. This model considers the number of atomic operations,
  the number of memory accesses and the number of accesses that can be
  served from the CPU cache.
\item Using this model, we show that the lower-latency memory accesses
  provided by PIM are \emph{not} sufficient for sequential PIM data
  structures to outperform efficient traditional concurrent data
  structures.
 \item We propose new designs for PIM data structures using techniques
   such as combining, partitioning and pipelining.  Our evaluations
   show that these new PIM data structures can outperform traditional
   concurrent data structures, with a significantly simpler design.
\end{itemize}

The paper is organized as follows. In
Section~\ref{section:hardware_model}, we briefly describe our
assumptions about the hardware architecture.  In
Section~\ref{section:performance_model}, we introduce a simplified
performance model that we use throughout this paper to estimate the
performance of our data structures using the hardware architecture
described in Section~\ref{section:hardware_model}.  In
Sections~\ref{section:pointer_chasing} and \ref{section:contended}, we
describe and analyze our PIM data structures and use our model to
compare them to prior work.  We also use current DRAM architectures to
simulate the behavior of our data structures and evaluate them
compared to state-of-the-art concurrent data structures.  Finally, we
present related work in Section~\ref{section:related_work} and
conclude in Section~\ref{section:conclusion}.

