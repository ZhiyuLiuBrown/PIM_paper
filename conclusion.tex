\section{Conclusion}
\label{section:conclusion}
In this paper, we study the problem of exploiting the low memory access latency of PIM memory to design efficient concurrent data structures with the PIM memory. 
To analyze and compare performance of our PIM-managed data structures and concurrent data structures in the literature, we propose a simplified performance model based on the latency numbers in prior work on PIM
memory and on evaluations of operations in multiprocessor architectures. 
We show that certain naive PIM data structures cannot outperform traditional concurrent data structures, 
due to lack of parallelism, or high communication costs between CPUs and PIM cores.  
To improve performance of PIM data structures, we propose novel designs for different types of data structures with different optimization techniques.   
More specifically, we present a PIM-managed linked-list with combining optimization and a skip-list with partitioning optimization, as representatives of pointer-chasing data structures, 
and a PIM-managed FIFO queue with pipelining optimization, as a representative of high-contention data structures. 
We prove that those simple PIM algorithms can beat state-of-the-art concurrent data structures, making PIM memory a promising platform for designing concurrent data structures.
 