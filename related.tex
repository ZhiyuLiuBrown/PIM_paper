\section{Related Work}

Researchers have studied the PIM model for decades (e.g., \cite{Stone1970, Kogge1994, 
Gokhale1995, Patterson1997, Oskin1998, KangHYKGLTP99, Hall1999}). 
Implementations of PIM memory have become much more feasible recently due to the advancements 
in 3D-stacked technology that can stack memory dies on top of a logic layer 
\cite{jeddeloh2012, Loh2008, Black2006}. 
Based on this technology, Micron and other vendors together implemented a prototype of 
PIM memory called the Hybrid Memory Cube \cite{website:HMC} a few years ago. 
Since then, the PIM model has drawn a lot of attention in the computer architecture community. 
Different PIM-based architectures have been proposed, either for general purposes or for 
specific applications \cite{Ahn2015:1, Ahn2015:2, Zhang2014:TTP, hsieh2016accelerating,
Azarkhish16, Akin2015:DRM, Azarkhish2015, AzarkhishPRLB17, boroumand2016, ZhuASSHPF13, ZhuGSPF13}.

Besides PIM memory's low energy consumption and high bandwidth 
(e.g., \cite{Ahn2015:2, Zhang2014:TTP, ZhuASSHPF13, AzarkhishPRLB17}), 
researchers have also explored the benefits of PIM memory's low memory access latency
\cite{Loh2008, hsieh2016accelerating, Azarkhish16}, which is what we focus on in this paper. 
To our knowledge, however, we are the first to utilize PIM memory for designing efficient 
concurrent data structures. 
Although some researchers have studied how PIM memory can help speed up concurrent 
operations to data structures, such as parallel graph processing \cite{Ahn2015:2} and  
parallel pointer chasing on linked data structures \cite{hsieh2016accelerating}, 
the applications they consider require very simple, if any, synchronization between operations. 
In contract, operations to concurrent data structures can interleave in arbitrary orders, 
and therefore they have to correctly synchronize with one another in all possible situations. 
This makes designing concurrent data structures with correctness guarantees like 
linearizability \cite{Herlihy90} very challenging. 

Moreover, no one has ever compared the performance of data structures in the PIM model 
with that of existing concurrent data structures in the classic shared memory model. 
We analyze and evaluate concurrent linked-lists and skip-lists, 
as representatives of pointer-chasing data structures, and concurrent FIFO queues, 
as representatives of contended data structures.
For linked-lists, we compare our PIM-managed implementation with the concurrent linked-list with 
fine-grained locks \cite{Heller05}, and the one implemented using flat combining 
\cite{Hendler10}, a generic technique to design concurrent data structures.  
For skip-lists, we compare our implementation with the lock-free skip-list \cite{Herlihy08} 
and a skip-list with flat combining and partitioning optimization. 
For FIFO queues, we compare our implementation with the flat-combining FIFO queue 
\cite{Hendler10} and the F\&A-based FIFO queue \cite{Morrison13}.  
As we will show in the paper, our simple PIM-managed concurrent data structures can in theory 
outperform those concurrent data structures, 
making PIM memory a promising platform to run concurrent data structures.