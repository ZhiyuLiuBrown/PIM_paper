\section{Related Work}
\label{section:related_work}
The PIM model is undergoing a renaissance. 
Studied for decades 
(e.g., \cite{Stone1970, Kogge1994, Gokhale1995, Patterson1997, Oskin1998, KangHYKGLTP99, Hall1999}), 
this model has recently re-emerged due to advances in 3D-stacked techology that 
can stack memory dies on top of a logic layer \cite{jeddeloh2012, Loh2008, Black2006, Kim2014HotChip}. 
For example, Micron and others have recently released a PIM prototype called 
the Hybrid Memory Cube \cite{website:HMC}, and the model has again become the focus of architectural research.
Different PIM-based architectures have been proposed, either for general purposes or for 
specific applications \cite{Ahn2015:1, Ahn2015:2, Zhang2014:TTP, hsieh2016accelerating,
Azarkhish16, Akin2015:DRM, Azarkhish2015, AzarkhishPRLB17, boroumand2016, ZhuASSHPF13, ZhuGSPF13}.

The PIM model has many advantages, including low energy consumption and high bandwidth 
(e.g., \cite{Ahn2015:2, Zhang2014:TTP, ZhuASSHPF13, AzarkhishPRLB17}). 
Here, we focus on one more: low memory access latency 
\cite{Loh2008, hsieh2016accelerating, Azarkhish16}.
To our knowledge, however, we are the first to utilize PIM memory for designing efficient 
concurrent data structures. 
Although some researchers have studied how PIM memory can help speed up concurrent 
operations to data structures, such as parallel graph processing \cite{Ahn2015:2} and  
parallel pointer chasing on linked data structures \cite{hsieh2016accelerating}, 
the applications they consider require very simple, if any, synchronization between operations. 
In contrast, operations to concurrent data structures can interleave in arbitrary orders, 
and therefore have to correctly synchronize with one another in all possible situations. 
This makes designing concurrent data structures with correctness guarantees like 
linearizability \cite{Herlihy90} very challenging. 

Moreover, no one has ever compared the performance of data structures in the PIM model 
with that of state-of-the-art concurrent data structures in the classic shared memory model. 
We analyze and evaluate concurrent linked-lists and skip-lists, 
as representatives of pointer-chasing data structures, and concurrent FIFO queues, 
as representatives of contended data structures.
For linked-lists, we compare our PIM-managed implementation with well-known approaches 
such as fine-grained locking \cite{Heller05} and flat combining 
\cite{Hendler10, Fatourou12, Hendler:2010:DISC}.

For skip-lists, we compare our implementation with the lock-free skip-list \cite{Herlihy08} 
and a skip-list with flat combining and partitioning optimization. 
For FIFO queues, we compare our implementation with the flat-combining FIFO queue 
\cite{Hendler10} and the F\&A-based FIFO queue \cite{Morrison13}. 