%\section{Processing in Memory}
%\label{section:model}
\section{Hardware Architecture}


%\subsection{Hardware model}
\label{section:hardware_model}

In an example architecture utilizing PIM memory \cite{Ahn2015:2, Zhang2014:TTP, Ahn2015:1, boroumand2016}, 
multiple CPUs are connected to the main
memory, via a shared crossbar network, as illustrated in Figure \ref{figure:model}.
The main memory consists of two parts---one is a standard DRAM accessible by CPUs, 
and the other, called the \textit{PIM memory}, is divided into multiple partitions, 
called \textit{PIM vaults} or simply vaults.  
According to the \textit{Hybrid Memory Cube (HMC)} specification 1.0 \cite{website:HMC}, 
each HMC consists of 16 or 
32 vaults and has a total size of 2GB or 4 GB (so each vault's size is roughly 100MB).\footnote{
These small sizes are preliminary, and it is expected that each vault will become larger when the 
PIM memory will be commercialized.} 
We assume the same specifications in our PIM model, although the size of the PIM memory and 
the number of its vaults can be bigger. 
Each CPU core also has access to a hierarchy of L1 and L2 caches backed by DRAM,
and a last level cache shared among multiple cores. 

\begin{figure}[ht!]
%$\hrulefill$
%\\
%\\
\centering
\includegraphics[width=.6\linewidth]{model.eps}
%$\hrulefill$
\caption{An example PIM architecture}
\label{figure:model}
\end{figure}

Each vault has a \textit{PIM core} directly attached to it.
We say a vault is \textit{local} to the PIM core attached to it, and vice versa.
A PIM core is a lightweight CPU that may be slower than a full-fledged CPU
with respect to computation speed \cite{Ahn2015:2}. 
A PIM core can be thought of as an in-order CPU with a small private L1 cache.
A vault can be accessed only by its local PIM core.\footnote{
Alternatively, we could assume that a PIM core has direct access to the remote vaults, but slower than to the local vault.}
Recent work proposes efficient cache coherence mechanisms between PIM cores and CPUs  
(e.g., \cite{boroumand2016, Ahn2015:1}), but this introduces additional complexity. 
We show that we can design efficient concurrent PIM data structures even if there is no coherence.
Although a PIM core has lower performance than a state-of-the-art CPU core, 
it has fast access to its local vault.

A PIM core communicates with other PIM cores and CPUs via messages.
Each PIM core, as well as each CPU, has buffers for storing incoming messages.
A message is guaranteed to eventually arrive at the buffer of its receiver.
Messages from the same sender to the same receiver are delivered in FIFO order: 
the message sent first arrives at the receiver first. 
However, messages from different senders or to different receivers can arrive in an arbitrary order. 

We assume that a PIM core can only perform read and write operations 
to its local vault, while a CPU also supports more powerful atomic operations, such as \emph{Compare-And-Swap (CAS)} 
and \emph{Fetch-And-Increment (F$\&$A)}.
Virtual memory can be realized efficiently if  
each PIM core maintains its own page table for the local vault~\cite{hsieh2016accelerating}.